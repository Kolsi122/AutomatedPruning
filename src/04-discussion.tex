\section{Conclusions and Future Work}
We have introduced an automated method for simulation of pruning of
trees and tree colonies. The objective was to propose pruning that
maximizes light exposure of buds within the crown and we used two step
method, where first step prunes the tree to a desired shape, and the
second step maximizes the bud irradiance. Our results show that this
combination comes close to the human pruning regarding the light
distribution inside the tree crown. The pruning simulation of a group of
trees for several consecutive years showed that the method could also be
used for the tree training towards desirable growing form. Since DDE is
a stochastic method, the set of excess branches that have to be removed
can vary from simulation to simulation. The solutions are not always
equally successful with respect to tree light crown distribution, but
the same is also true by the pruning solutions proposed by human pruning
expert.

The possible limitation of the proposed method is its strong dependency
on the tree growth model topological structure. This is important
because the challenging step in the automated tree pruning is the
construction of the appropriate tree model from tree images. Successful
algorithms of this kind are presented in {[}1, 5, 10{]}. However, they
are not directly compatible with the EduAPPLE tree growth model and a
reconstruction algorithm should be developed for that purpose. Since the
proposed pruning method shows pruning results comparable to human
expert, the construction of such an algorithm would be desirable. Some
work in this direction has already been done \cite{kohek_estimation_2017}, but it is too
early to assess its efficiency.

The main contribution of our work is to show that good pruning results
can be obtained automatically and without a fixed set of pruning rules.
This is not surprising since the pruning rules were developed by the
long-term experiments whose goal was to improve yield quality as well as
its quantity. Once the main parameters that influence the yield were
determined, the pruning techniques have been developed that maximize the
influence of those parameters and that is what the DDE objective
function is trying to model. Since the light exposure is a critical
factor, the proposed method searched for the combination of cuts that
maximize the light exposure. This resulted in the higher light
distribution inside the tree crown. This distribution was also used for
evaluation of tree pruning since the obtained tree forms cannot be
compared directly. While the shapes of the trees in Fig.~\ref{fig:my_figure4} differs much
their light distributions are comparable. The additional pruning
benchmark is the number of internodes, which represents the equivalent
to the tree volume. The amount of removed internodes corresponds to the
pruning intensity. Fig.~\ref{fig:my_figure5}~b shows that the human expert preferred
slightly less aggressive pruning than the DDE method. The pruning
intensity by the proposed automated pruning is controlled by the
relation between tree volume and the value of the \(s_{\mathrm{\max}}\)
parameter. In order to preserve the tree pruning intensity, as the tree
is growing, the value of \(s_{\mathrm{\max}}\) have to increase with the
increasing number of internodes in the tree. In our experiments, 20
turned out to be the good initial value for the \(s_{\mathrm{\max}}\)
parameter.
