\begin{abstract}
Tree pruning is labor and cost intensive task, but it is a necessary
measure that ensures high yield of good quality in horticulture and increases the overall health of trees in general. However, a great deal of experience is necessary in order to correctly prune a tree 
and a serious damage can be incurred if pruning is applied incorrectly.
An extensive research has been made on how to automate the pruning
procedure and what is the actual effect of pruning on trees. 

We introduce a two-step method for automatic tree pruning that simulates pruning by detecting branches that should be removed in order to optimize a virtual tree light intake. Our two step algorithm starts by trimming the tree is into a desired initial shape. In the second step the discrete differential evolution method is used to optimize the light distribution within the tree crown by detecting branches that should be trimmed, and virtually removing them. Our algorithm does not use a  predefined set of pruning rules. Instead, it is an automatic optimization driven by the discrete differential evolution algorithm. We demonstrate our method by simulated pruning of virtual trees by using the EduAPPLE algorithm and we compare the pruning results to human expert. Our comparison shows an agreement with the expert evaluation and the trees can be formed into various growing forms, depending on the initial shape used. We believe that our algorithm is an important step towards pruning task automation.
\end{abstract} 
