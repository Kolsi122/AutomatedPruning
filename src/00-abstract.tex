\begin{abstract}
Tree pruning is a labor and cost intensive task, but it is a necessary
measure that ensures high yield of good quality in horticulture and increases the overall health of trees in general. However, a great deal of experience is necessary in order to correctly prune a tree without causing series damage to it.
Extensive research has been made on how to automate the pruning
procedure and what is the actual effect of pruning on trees. 

We introduce a novel two-step algorithm for automatic tree pruning that simulates pruning by detecting branches that should be removed in order to optimize a virtual tree light intake and maintain distance of the trees. Our two step algorithm starts by trimming the tree into a desired initial shape. In the second step a discrete differential evolution method is used to optimize the light distribution within the tree crown by detecting branches that should be trimmed, and virtually removing them. Our algorithm does not use a predefined set of pruning rules. Instead, it is an automatic optimization driven by the discrete differential evolution algorithm. We demonstrate our method by simulating the pruning of virtual trees in the EduAPPLE system by our new algorithm and show that it provides results comparable to the results obtained by a human expert. We believe that our algorithm is an important step towards pruning task automation.
\end{abstract} 
