\begin{abstract}


Tree pruning is labor and cost intensive task, but it is a necessary
measure that ensures high yield of good quality in horticulture. An
extensive research has been made on how to automate the tedious pruning
procedure. We present a new two-step method that detects branches that
should be removed: an important part of pruning task automation. Our two
step algorithm start by trimming the tree is into a desired initial
shape. In the second step the discrete differential evolution is used to
optimize the light distribution within the tree crown. Our algorithm
does not depend on any predefined set of pruning rules. The method has
been used for pruning virtual trees inside the teaching tool EduAPPLE,
and the pruning results were compared to human expert. Our comparison
shows an agreement with the expert evaluation and the trees can be
formed into various growing forms, depending on the initial shape used.



\end{abstract} 
