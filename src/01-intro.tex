\section{Introduction}
Tree pruning is the process of removing branches usually with the
objective of allowing more tree in the canopy. Old and dead branches are
removed as well as living ones to balance the reproductive and
vegetative growth in the fruit production. Pruning is one of the most
expensive and labor-intensive tasks that is responsible for
approximately 20\% of the annual pre-harvest cost for crops like apples,
cherries, and pears \cite{karkee_identification_2014}. A large crew of trained seasonal workers is
needed to accomplish this task each winter, following a set of
predefined rules \cite{akbar_novel_2016} . Extensive research has been done on how to
automate the tree pruning process to reduce costs and/or demands for the
skilled workforce \cite{karkee_identification_2014,moore_mechanical_1958,jensen_effects_1980}
on mass pruning by maintaining the specified distance from the tree
canopy center with limited ability to ensure a high pruning quality. The
results were not satisfying, as the evidenced by reduced quality and
yield of fruit \cite{karkee_identification_2014}. The solution to that problem is the pruning
system, capable of identifying the individual branches that have to be
removed.

Some important work has been done in this direction already, where a
mobile platform for automated pruning of grape vines has been introduced
\cite{botterill_robot_2017}. Although the apple tree structure is more complex that those of
grape vines, the essence of the approach to the pruning automation is
similar. A 3D model of the vines is generated by using computer vision
that recognized the tree structure and an AI- based system decides which
canes to prune. Actual pruning is carried out by a six degree-of-freedom
robotic arm. Both, the apple tree and vine pruning are carried out while
the plants are in a dormant state, and in both cases, a set of
predefined pruning rules controls the pruning.

One of the bottlenecks of the current models is the 3D tree model
reconstruction and many approaches have been introduced~\cite{,livny_automatic_2010,xie_tree_2016,zhang_data-driven_2014}. By
applying the tree pruning rules on the generated tree 3D model the
surplus branches are identified that have to be removed~\cite{akbar_novel_2016,elfiky_automation_2015,medeiros_modeling_2017}.
These rules aim to increase the light input into the tree crown and thus
to improve the tree health \cite{simon_does_2006} and, in effect, fruit quality
\cite{bastias_light_2012}. The overall fruit quality is additionally enhanced by removing
weak shoots, and thus controlling the tree size as well.

One of the main problems of the previous work is that the trees are
removed by using fixed rules. In this paper, we propose an alternative
to the fixed set of pruning rules, by introducing a new two-step
procedure, which combines the pruning approach of the first mechanical
pruning systems with the selective pruning. In the first step, the tree
is trimmed according to the predefined template to maintain a desired
tree height and the distance to its neighbors. In the second step,
discrete differential evolution (DE) \cite{strnad_novel_2017} is used to identify the
individual branches, that should be removed. We show our results on an
experiment, where we used two different pruning templates in the first
step of pruning, a cylinder and a cone. We have implemented our method
in EduAPPLE \cite{kohek_eduapple:_2015} that is a 3D tree model simulator. We compared the
light distribution inside the tree crown after pruning trees using the
proposed method with those pruned by the expert and the results were
comparable. Using the proposed pruning method for a few consecutive
years inside EduAPPLE preserved the distances between the trees, along
with their height, which is essential in high-density orchards.
