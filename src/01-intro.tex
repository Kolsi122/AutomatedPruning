\section{Introduction}
Tree pruning is the process of removing branches usually with the main
objective of allowing more light in the canopy. Old and dead branches are
removed as well as living ones to balance the reproductive and
vegetative growth in the fruit production. \bb{Pruning must be done carefully not to damage the tree, to prevent fungus infections on the cuts, and not to damage the tree in general. This makes pruning} one of the most
expensive and labor-intensive tasks that is responsible for
approximately 20\% of the annual pre-harvest cost for crops like apples,
cherries, and pears~\cite{karkee_identification_2014}. A large crew of trained seasonal workers is needed to accomplish this task each winter, following a set of predefined rules~\cite{akbar_novel_2016}. 

Extensive research has been done on how to automate the tree pruning process \bb{of real plants} in order to reduce costs and/or demands for the skilled workforce~\cite{jensen_effects_1980,karkee_identification_2014,moore_mechanical_1958}
on mass pruning by maintaining the specified distance from the tree
canopy center with limited ability to ensure a high pruning quality. The
fully automated results were not satisfying, as evidenced by the reduced quality and
yield of fruit~\cite{karkee_identification_2014}. 
Another previous work used mobile platform for automated pruning of grape vines~\cite{botterill_robot_2017}. 
Although the tree structure is more complex that those of
grape vines, the essence of the approach to the pruning automation is
similar. Both approaches create a 3D model of the plant by using computer vision
that recognized the tree structure and a decision system determines which
parts of the plant should be removed. Actual pruning is carried out by a six degree-of-freedom
robotic arm. Both, the apple tree and vine pruning are carried out while
the plants are in a dormant state, and in both cases, a set of
predefined pruning rules controls the pruning.

One of the bottlenecks of the current models is the 3D tree model
reconstruction. \bb{Plant reconstruction is an open problem by itself}
and many approaches have been introduced~\cite{livny_automatic_2010,xie_tree_2016,zhang_data-driven_2014}. 
By applying the tree pruning rules on the generated tree 3D model the
surplus branches are identified that have to be removed~\cite{akbar_novel_2016,elfiky_automation_2015,medeiros_modeling_2017}.
\bb{As in pruning in general,} these rules aim to increase the \bb{irradiance intake of the canopy},  to improve the tree health~\cite{simon_does_2006} and, in effect, fruit quality~\cite{bastias_light_2012}. 
The overall fruit quality is additionally enhanced by removing
weak shoots, and thus controlling the tree size.

One of the main problems of the previous work is that the trees are
removed by using fixed rules. \bb{The key observation of our work is
that we can generate the rules for each plant individually by using mathematical optimization.}
In this paper, we propose an alternative
to the fixed set of pruning rules, by introducing a new two-step
procedure, which combines the pruning approach of the first mechanical
pruning systems with the selective pruning. In the first step, the tree
is trimmed according to a predefined template to maintain a desired
tree height and the distance to its neighbors. \bb{The first step attempts to minimize mutual tree shading, bud drying by their abrasion, and mutual tree competition.}
In the second step, \bb{a mathematical optimization algorithm is used to determine which branch should be removed. Although many options exist, we use }
the discrete differential evolution (DE)~\cite{strnad_novel_2017}. 

We have implemented our method in \bb{a softare apple tree plant simulator} EduAPPLE~\cite{kohek_eduapple:_2015}. We run several experiments,
where we used two different pruning templates in the first
step of pruning, a cylinder and a cone. 
We compared the light distribution inside the tree crown after pruning trees using the
\bb{newly-developed} method with those pruned by the expert.
Our results show \bb{comparable levels of irradiance withing the canopy.}
Moreover, by using the proposed pruning method for several consecutive
years, \bb{the next sentence is weird - trees cannot walk or move...} 
the trees automatically preserved their distances, along
with their height, which is essential in high-density orchards.
