\section{Introduction}
The tree canopy pruning is the process of removing branches usually with the main
objective of allowing more light in the tree crown. Living branches are
removed as well as old and dead ones to balance the reproductive and
vegetative growth in the wood and fruit production. Pruning must be done carefully not to damage the tree and to prevent fungus infections on the cuts. This makes pruning one of the most expensive and labor intensive tasks that is responsible for approximately 20\% of the annual pre-harvest cost for crops like apples, cherries, and pears~\cite{karkee_identification_2014}. A large crew of trained seasonal workers is needed to accomplish this task, following a set of predefined rules~\cite{akbar_novel_2016}. 

Extensive research has been done on how to automate the tree pruning process in order to reduce costs and/or demands for a skilled workforce~\cite{jensen_effects_1980,karkee_identification_2014,moore_mechanical_1958}. This prior research has resulted in the development of early mechanical systems for mass pruning which mainly maintain the specified distance of the tips of the branches from the tree canopy center. The
fully automated results were not satisfactory, as evidenced by the reduced quality and
yield of fruit~\cite{karkee_identification_2014}. 
Another previous work used mobile platform for automated pruning of grape vines~\cite{botterill_robot_2017}. 
Although the structure of tree is more complex that those of
grape vines, the essence of the approach to pruning automation is
similar. Both approaches create a 3D model of the plant by using computer vision
that recognized the tree structure and a decision system determines which
parts of the plant should be removed. Actual pruning is carried out by a six degrees-of-freedom
robotic arm. Both the apple tree and grape vine pruning are carried out while
the plants are in a dormant state, and in both cases, a set of
predefined pruning rules controls the pruning, e.g., various criteria for removing branches based or length, or removing based on start and/or end point spacing.

In those approaches, the pruning is carried out in two phases. First the 3D model of a tree is generated upon which the pruning rules are applied. Plant reconstruction is an hard problem by itself and many approaches have been introduced~\cite{livny_automatic_2010,xie_tree_2016,zhang_data-driven_2014}. 
By applying the tree pruning rules on the generated tree 3D model the
surplus branches are identified that have to be removed~\cite{akbar_novel_2016,elfiky_automation_2015,medeiros_modeling_2017}.
The pruning rules aim to increase the irradiance intake of the canopy,  to improve the tree health~\cite{simon_does_2006} and, in effect, fruit quality~\cite{bastias_light_2012}. 


One of the main problems of the previous work is that the branches are
removed by using fixed rules. The key observation of our work is
that we can generate rules for each plant individually by using mathematical optimization.
In this paper, we propose an alternative
to the fixed set of pruning rules by introducing a new two-step
procedure which combines the pruning approach of the existing mechanical
pruning systems with the selective pruning. In the first step, the tree
is trimmed according to a predefined template to maintain a desired
tree height and distance to its neighbors. The first step attempts to minimize mutual tree shading, bud drying from abrasion, and mutual tree competition.
In the second step, a mathematical optimization algorithm is used to determine which branches should be removed. Although many options exist, we use 
the discrete differential evolution (DDE)~\cite{strnad_novel_2017}. 

We have implemented our method in the software apple tree plant simulator EduAPPLE~\cite{kohek_eduapple:_2015}. We ran several experiments using two different pruning templates in the first
step of pruning: a cylinder and a cone. 
We compared the light distribution inside the tree canopy after pruning trees using the
newly-developed method with those pruned by the expert.
Our results show comparable levels of irradiance within the canopy.
Additionally, by using the proposed pruning method for several consecutive
years, the space between trees was kept free of competing branches, along with preserving their height, which is essential in high-density orchards.
