\section{Materials and Methods}

In the proposed approach to automated tree pruning, a predetermined set
of tree pruning rules is replaced by an objective function which
includes the goals that have to be achieved by the pruning, combined
with the pre-pruning of a tree to cylindrical/conical shape. For the
sake of this experiment, the objective function prefers the maximization
of received light in the tree crown. For the optimization of the
objective function, discrete differential evolution method is used,
closely connected to a virtual tree model \cite{kohek_eduapple:_2015}, which is based on the
concept of self-organization \cite{palubicki_self-organizing_2009}.

\subsection{Tree Growth Model}

A tree model used in our approach is represented as a hierarchy of
modules as shown in Fig.~\ref{fig:my_figure1}~a) that is a common approach used in many plant
simulators such as \cite{de_reffye_plant_1988,palubicki_self-organizing_2009,pirk_plastic_2012,prusinkiewicz_development_1988,stava_inverse_2014}. The point at which one or more
leaves are attached to the stem is called a node. The part of the stem
between two nodes is an internode. The growth is controlled by apical
meristem that is a region of dividing cells that responds to gravity and
to the incoming light. In this way the tree grows against gravity
(gravitropism) and seeks the light (phototropism). Depending on the
plant DNA and environmental factors (light, temperature, nutrients) the
plant produces lateral buds that are either dormant or active. An
internode with attached leaves and lateral bud is called a metamer and a
sequence of metamers created at single spurt forms a shoot. The shoot
axis is produced by the terminal bud which is located at the end of the
shoot.

Our approach uses the previously developed framework called EduAPPLE~\cite{kohek_eduapple:_2015}, 
where the tree growth is driven by buds' illuminations~\cite{benes_efficient_1996,benes_visual_1997,mech_visual_1996} and a competitive process for growth resources \cite{alsweis_modeling_2005,arvo_modeling_1988,palubicki_self-organizing_2009,runions_modeling_2007}, which leads to the self-organizing structure of the tree. The
tree attempts to maximize its branch mass by growth and light intake by
growth. Buds with higher irradiance produce new shoots that fill the
empty space. Buds in lower irradiance produce quickly growing shoots
that attempt to get from shade. Poorly illuminated buds do not create
new shoots, become dormant, and can form new shoots later, when the
conditions are more favorable. The key factor of this simulation is the
calculation of the illumination of leaves that feed buds and several
algorithms exists. Although inner reflection can be considered \cite{soler_efficient_2003},
the algorithms are usually time consuming, and the indirect light does
not contribute significantly to growth, because major light intake is
given by the direct lighting. A faster way is to use only the direct
irradiance~\cite{benes_visual_1997,benes_efficient_1996,mech_visual_1996,pirk_plastic_2012}. 
The tree cast shadow on itself and thus
forming a shadow space of the tree as shown in Fig.~\ref{fig:my_figure1}~b). The shadow
density is highest at the trunk; thus no leaves/shoots can grow from the
dormant buds positioned there. The key contribution of pruning is in
improving the illumination of the inner parts of the tree. When the
lighting conditions around the stem improve after removing some
branches, dormant buds can reactivate and start to produce new shoots.
\begin{figure}[hbt]
    \centering
    \includegraphics[width=4.5in]{figs/image1.jpg}
    \caption{a) Plant modules of a self-organizing tree growth
model, b) shadow space of a tree in used tree growth model from \cite{kohek_eduapple:_2015}.}
    \label{fig:my_figure1}
\end{figure}


\subsection{Illumination}

We calculated the inner shading by using the algorithm from \cite{palubicki_self-organizing_2009}
that was later extended in~\cite{pirk_plastic_2012,stava_inverse_2014,strnad_novel_2017}. The light distribution
inside the tree crown is calculated from the illuminated leaves that are
sources of a conical shadow volume. We calculate the bud illumination.
Suppose, we have a bud inside one of the shadow volumes of one of the
neighbors. Let \(d_{y}\) be the bud's vertical distance from the volume
apex and \(d_{x,z}\) its horizontal distance from the volume axis, then
the received shadow, originating from a given shadow volume is
calculated by \cite{strnad_novel_2017}:
\begin{equation}
\Delta s = \left\{ \begin{matrix}
\text{ab}^{- 0.8\left( d_{y} + d_{x,y} \right)}, & \mathrm{\text{if}}\ d_{x,y} < d_{y} \\
0, & \mathrm{otherwise} \\
\end{matrix},
\end{equation}
where \(a\) \textgreater{} 0 and \(b\)\textgreater{} 0 are model
parameters from \cite{palubicki_self-organizing_2009} and were set to \(a = 0.05\) and \(b = 2\).
Total irradiance of a given bud is then calculated by:
\begin{equation}
  Q = max\begin{Bmatrix}
1 - s, & 0 \\
\end{Bmatrix},  
\end{equation}
where \(s\) is the cumulative contribution of all shadow volumes,
captured by the given bud.

\subsection{Differential Evolution and Pruning}
Differential Evolution (DE) is a heuristic approach for minimizing
possibly nonlinear and non-differential continuous objective function.
Developed by Storm and Price \cite{storn_differential_1997}, the method aims to optimize
certain properties of a system pertinently choosing the system
parameters, usually represented as a vector. Objective function models
the objectives while incorporating potential constraints. DE employs
evolutionary operators like mutation, crossover, and selection.

Discrete DE (DDE) variants have been presented in relation to specific
combinatorial optimization problems, e.g., \cite{davendra_flow_2009,pan_discrete_2008,wang_novel_2010}. The DDE has
been used for the optimization of light condition inside the tree crown
by removing certain branches in \cite{strnad_novel_2017} in particular four optimization
methods have been used: BinDE, IndexDE, PathDE, and SetDE. Through
optimization of pruning locations, a combination of cuts is obtained
that maximizes the amount of light received by the remaining buds of the
tree crown. For that purpose, for all buds in the tree crown the
irradiance is calculated by the use of Eq. (2). Bud's irradiance, \(Q\)
corresponds to the percentage of available light intercepted by the bud.
For the sake of faster detection of ancestor/successor relationship
between internodes, each internode is associated with a unique
variable-length binary string as shown in~Fig.~\ref{fig:my_figure1}~b. Tree crown light
distribution is calculated next, where the irradiance of each bud in the
crown is assigned to one of ten quantization classes of equal width on
the interval {[}0, 1{]}. The objective function, \emph{f}(\textbf{x}),
is:
\begin{equation}
 f\left( \mathbf{x} \right) = \frac{\sqrt{S_{\mathrm{\text{tree}}}}}{H}\sum_{i = 1}^{10}{i \times h_{i}}, 
\end{equation}
where \(S_{\mathrm{\text{tree}}}\) is the number of remaining internodes
after pruning, \emph{H} is the total number of buds, and \(h_{i}\) is
the number of buds in the \emph{i}-th class of light distribution.

We define an optimization function that describes the light interception
of the tree as the sum of interception of all buds
\begin{equation}
   f\left( \mathbf{x} \right) = \sum_{i = 1}^{n}E_{i},
\end{equation}
where $n$ is the number of buds that remain after the pruning,
$E_{i}$ is the irradiance of the \(i\)-th bud. The function $f(\mathbf{x})$ is the total irradiance of the tree for a configuration \(x\) of the buds,
also called the solution vector. The solution vector is denoted by
\(\mathbf{x} = \ \left\{ x_{1},\ \ldots,x_{s} \right\}\) and it contains
the encoded sequence of cut positions, labeled by the corresponding
internodes. The root is denoted by a bit-string ``0''. A `0' or `1' is
appended to the parent's string for each main or lateral child
internode, respectively (Fig.~\ref{fig:my_figure1}~b). Each cut position \(x_{i}\)
identifies the internode, at which the branch is removed. Variable
\emph{s} in this case is a population size and is a value between
\(s_{\mathrm{\min}}\) and \(s_{\mathrm{\max}}\), which are custom set
parameters, representing the minimum and the maximum number of allowed
cuts. The objective function \(f(\mathbf{x})\) favors solutions that
provide a maximal improvement of light distribution with minimum amount
of removed biomass. It is very important to recognize a redundant cuts,
for example, if a cut should be made at internode ``00'', the cuts
``001'', ``000'', or ``0010'', would be redundant and thus unnecessary
as shown in Fig.~\ref{fig:my_figure2}. Our labeling allows for quick detection of the
redundant cuts. In a given example the later three internodes share the
same prefix ``00'', which is the label of their predecessor and with
removing it, all of its successors are removed as well.

\begin{figure}[hbt]
    \centering
    \includegraphics[width=3.3in]{figs/image2.jpeg}
    \caption{Mapping of the solution vertex into the tree cutting
sequence. Solution vector components that exceed the threshold are
converted into the cuts with the help of the cut list.}
    \label{fig:my_figure2}
\end{figure}


For our experiment, we have selected only BinDE optimization method
because the pruning results were visually close to those of the human
expert. The encoding scheme of BinDE uses the real-valued vectors to
represent solution genotypes. For the mapping of genotypes to pruning
instances, the intermediate cut list is used. It is produced by
traveling the tree model in a depth-first manner and recording the
labels of all internodes encountered in the process into a list. The
length of the cut list defines the dimensions of solution vectors. Each
component \(x_{i,j} \in \ \left\lbrack 0,\ 1 \right\rbrack\) of the
solution vector \(\mathbf{x}_{i}\ \)determines whether \emph{j}-th
internode from the cut list will be selected as the cutting point or
not. For that purpose the binary vector \(\mathbf{z}_{i}\) is
constructed by thresholding:
\begin{equation}
    z_{i,j} = \ \left\{ \begin{matrix}
0, & \mathrm{\text{if}}\ x_{i,j} < 0.5 \\
1, & \mathrm{\text{otherwise}} \\
\end{matrix}, \right.\
\end{equation}


If the number of cuts proposed by the vector \(\mathbf{z}_{i}\) violates
the solution size constraints, the threshold is adjusted up or down
until the constraints are met. The entire mapping operation is shown in
Fig.~\ref{fig:my_figure2}.

\subsection{Tree Height and Neighboring Distance Control}
Although the DDE method significantly improves the light conditions
inside the tree crown, it offers no control over the tree size or the
distance to neighbor trees. Since the modern orchards are mostly
protected by the anti-hail nets, it is crucial to keep apple trees at a
certain height. Similarly, the neighboring distance that, has to be
preserved during the entire lifetime of an orchard.

A straightforward method to get control over the tree height and
neighboring distance would be the extension of the objective function
\(f\left( \mathbf{x} \right)\) with additional constraints regarding the
tree size. It turned out, however, that while the integration of the
tree height into the tree model and objective function is possible, the
determination of the extent of the tree crown after the pruning was not
very efficient.

Our inspiration for the solution to this problem comes from observing a
human during manual pruning. Instead of just following the pruning
rules, the human pruner strives to shape the tree into one of the
well-defined growing forms. Those growing forms were developed over the
years by the experts and gave the best fruiting results under certain
growing conditions. In the high density orchards, the most common tree
forms are the Slender Spindle \cite{weber_optimizing_2000}, and more recent, the Tall
Spindle \cite{robinson_vision_2013}. While the first growing form is more conically shaped
the second resembles a cylinder. Tall spindle is very popular because it
is suitable for mechanical pruning and formation of the Fruiting Wall
\cite{robinson_vision_2013}.

Instead of changing the objective function, we propose adding a
preprocessing level to the DDE method. We call this step pre-pruning and
by adding this the desired form, the tree height and neighboring
distance can be controlled easily. We prune the trees to circular shapes
(cone and cylinder) with adjustable height and the base radius. After
the branches outside the chosen shape are trimmed off, the DDE method is
used to optimize the light condition inside the tree crown. The entire
tree pruning process is depicted in Fig.~\ref{fig:my_figure3}.
\begin{figure}[hbt]
    \centering
    \includegraphics[width=5.3in]{figs/image3.jpeg}
    \caption{Two-step virtual pruning process enables the control
over tree height and neighboring distance. First, the tree is pre-pruned
into a cone or cylinder shape with adjustable size. In the second step
the DDE method selectively removes branches to improve the light
conditions inside the tree crown.}
    \label{fig:my_figure3}
\end{figure}

The results of the final pruning are highly dependent on the maximum
allowed number of cuts \(s_{\mathrm{\max}}\), which has to be adjusted
to the tree age and complexity of the tree crown. In our implementation,
we set \(s_{\mathrm{\max}} = 20\) which provided good results for young
trees. While we used circular proxies in our work, other shapes could be
used: for example, orchards with Fruiting Wall planting system could use
rectangular blocks.
