\section{Conclusions}
Dormant tree pruning is very labor and cost intensive process, which is
essential for achieving good fruit yield and for the health of the fruit
trees. It is not surprising that extensive research has been done on how
to automate this process. Early experiments concentrated on the mass
pruning, which remains the most widespread automated pruning method to
this day. Although those methods produce good results on the Fruit Wall
planting system, they are not equally successful on other planting
systems, where more selective pruning is needed. In this article, we
presented a two-step method suitable for automated selective pruning.
Experimental pruning of virtual trees has shown, that the proposed
system is capable of achieving similar light condition in the three
crown as a human expert. The method is able to control the tree height
and the neighboring distance. In the performed simulations, it was
possible to shape the trees into desired growing form autonomously.
According to the results, the method has great potential. In order to be
used on real trees, some of technological problems have to be solved
first among which is the reliable reconstruction of a tree model out of
the set of photographs.
