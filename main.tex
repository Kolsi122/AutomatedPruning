\PassOptionsToPackage{unicode=true}{hyperref} % options for packages loaded elsewhere
\PassOptionsToPackage{hyphens}{url}
%

\documentclass[1p]{elsarticle}
%\documentclass[12pt,review,3p,authoryear,longnamesfirst]{elsarticle}

\usepackage{lmodern}
\usepackage{amssymb,amsmath}

  \usepackage[T1]{fontenc}
  \usepackage[utf8]{inputenc}
  \usepackage{textcomp} % provides euro and other symbols

% use microtype if available
\IfFileExists{microtype.sty}{%
\usepackage[]{microtype}
\UseMicrotypeSet[protrusion]{basicmath} % disable protrusion for tt fonts
}{}

\usepackage{hyperref}
\hypersetup{
            pdfborder={0 0 0},
            breaklinks=true}
\urlstyle{same}  % don't use monospace font for urls
\usepackage{longtable,booktabs}
% Fix footnotes in tables (requires footnote package)
\usepackage{graphicx,grffile}
\makeatletter

\makeatother

\date{}

%\usepackage{natbib}
%\bibliographystyle{unsrtnat}
\bibliographystyle{elsarticle-harv}

\begin{document}
\begin{frontmatter}


\title{An Algorithm for Automatic Dormant Tree Pruning }

\author[FERI]{Simon Kolmanič}\corref{CoresAuth}
\ead{simon.kolmanic@um.si}
\author[FERI]{Damjan Strnad}
\author[FERI]{\v{S}tefan Kohek}
\author[PURDUE]{Bedrich Benes}
\author[PURDUE]{Peter Hirst}
\author[FERI]{Borut v{Z}alik}

\address[FERI]{%Faculty of Electrical Engineering and Computer Science,\\
University of Maribor, Koroška cesta 46, 2000 Maribor, Slovenia}
\address[PURDUE]{Purdue University West Lafayette, IN 47906, USA}



\input{src\abstract}


\begin{keyword}
Computer Graphics \sep Virtual Tree  \sep Tree Pruning \sep Pruning Automation \sep Discrete Differential Evolution\sep Tree Height Control
\end{keyword}

\end{frontmatter}





\section{Introduction}

Tree pruning is the process of removing branches usually with the
objective of allowing more tree in the canopy. Old and dead branches are
removed as well as living ones to balance the reproductive and
vegetative growth in the fruit production. Pruning is one of the most
expensive and labor-intensive tasks that is responsible for
approximately 20\% of the annual pre-harvest cost for crops like apples,
cherries, and pears \cite{karkee_identification_2014}. A large crew of trained seasonal workers is
needed to accomplish this task each winter, following a set of
predefined rules \cite{akbar_novel_2016} . Extensive research has been done on how to
automate the tree pruning process to reduce costs and/or demands for the
skilled workforce \cite{karkee_identification_2014,moore_mechanical_1958,jensen_effects_1980}
on mass pruning by maintaining the specified distance from the tree
canopy center with limited ability to ensure a high pruning quality. The
results were not satisfying, as the evidenced by reduced quality and
yield of fruit \cite{karkee_identification_2014}. The solution to that problem is the pruning
system, capable of identifying the individual branches that have to be
removed.

Some important work has been done in this direction already, where a
mobile platform for automated pruning of grape vines has been introduced
\cite{botterill_robot_2017}. Although the apple tree structure is more complex that those of
grape vines, the essence of the approach to the pruning automation is
similar. A 3D model of the vines is generated by using computer vision
that recognized the tree structure and an AI- based system decides which
canes to prune. Actual pruning is carried out by a six degree-of-freedom
robotic arm. Both, the apple tree and vine pruning are carried out while
the plants are in a dormant state, and in both cases, a set of
predefined pruning rules controls the pruning.

One of the bottlenecks of the current models is the 3D tree model
reconstruction and many approaches have been introduced~\cite{,livny_automatic_2010,xie_tree_2016,zhang_data-driven_2014}. By
applying the tree pruning rules on the generated tree 3D model the
surplus branches are identified that have to be removed~\cite{akbar_novel_2016,elfiky_automation_2015,medeiros_modeling_2017}.
These rules aim to increase the light input into the tree crown and thus
to improve the tree health \cite{simon_does_2006} and, in effect, fruit quality
\cite{bastias_light_2012}. The overall fruit quality is additionally enhanced by removing
weak shoots, and thus controlling the tree size as well.

One of the main problems of the previous work is that the trees are
removed by using fixed rules. In this paper, we propose an alternative
to the fixed set of pruning rules, by introducing a new two-step
procedure, which combines the pruning approach of the first mechanical
pruning systems with the selective pruning. In the first step, the tree
is trimmed according to the predefined template to maintain a desired
tree height and the distance to its neighbors. In the second step,
discrete differential evolution (DE) \cite{strnad_novel_2017} is used to identify the
individual branches, that should be removed. We show our results on an
experiment, where we used two different pruning templates in the first
step of pruning, a cylinder and a cone. We have implemented our method
in EduAPPLE \cite{kohek_eduapple:_2015} that is a 3D tree model simulator. We compared the
light distribution inside the tree crown after pruning trees using the
proposed method with those pruned by the expert and the results were
comparable. Using the proposed pruning method for a few consecutive
years inside EduAPPLE preserved the distances between the trees, along
with their height, which is essential in high-density orchards.

\section{Materials and Methods}

In the proposed approach to automated tree pruning, a predetermined set
of tree pruning rules is replaced by an objective function which
includes the goals that have to be achieved by the pruning, combined
with the pre-pruning of a tree to cylindrical/conical shape. For the
sake of this experiment, the objective function prefers the maximization
of received light in the tree crown. For the optimization of the
objective function, discrete differential evolution method is used,
closely connected to a virtual tree model \cite{kohek_eduapple:_2015}, which is based on the
concept of self-organization \cite{palubicki_self-organizing_2009}.

\subsection{Tree Growth Model}

A tree model used in our approach is represented as a hierarchy of
modules as shown in Fig.~\ref{fig:my_figure1}~a) that is a common approach used in many plant
simulators such as \cite{de_reffye_plant_1988,palubicki_self-organizing_2009,pirk_plastic_2012,prusinkiewicz_development_1988,stava_inverse_2014}. The point at which one or more
leaves are attached to the stem is called a node. The part of the stem
between two nodes is an internode. The growth is controlled by apical
meristem that is a region of dividing cells that responds to gravity and
to the incoming light. In this way the tree grows against gravity
(gravitropism) and seeks the light (phototropism). Depending on the
plant DNA and environmental factors (light, temperature, nutrients) the
plant produces lateral buds that are either dormant or active. An
internode with attached leaves and lateral bud is called a metamer and a
sequence of metamers created at single spurt forms a shoot. The shoot
axis is produced by the terminal bud which is located at the end of the
shoot.

Our approach uses the previously developed framework called EduAPPLE~\cite{kohek_eduapple:_2015}, 
where the tree growth is driven by buds' illuminations~\cite{benes_efficient_1996,benes_visual_1997,mech_visual_1996} and a competitive process for growth resources \cite{alsweis_modeling_2005,arvo_modeling_1988,palubicki_self-organizing_2009,runions_modeling_2007}, which leads to the self-organizing structure of the tree. The
tree attempts to maximize its branch mass by growth and light intake by
growth. Buds with higher irradiance produce new shoots that fill the
empty space. Buds in lower irradiance produce quickly growing shoots
that attempt to get from shade. Poorly illuminated buds do not create
new shoots, become dormant, and can form new shoots later, when the
conditions are more favorable. The key factor of this simulation is the
calculation of the illumination of leaves that feed buds and several
algorithms exists. Although inner reflection can be considered \cite{soler_efficient_2003},
the algorithms are usually time consuming, and the indirect light does
not contribute significantly to growth, because major light intake is
given by the direct lighting. A faster way is to use only the direct
irradiance~\cite{benes_visual_1997,benes_efficient_1996,mech_visual_1996,pirk_plastic_2012}. 
The tree cast shadow on itself and thus
forming a shadow space of the tree as shown in Fig.~\ref{fig:my_figure1}~b). The shadow
density is highest at the trunk; thus no leaves/shoots can grow from the
dormant buds positioned there. The key contribution of pruning is in
improving the illumination of the inner parts of the tree. When the
lighting conditions around the stem improve after removing some
branches, dormant buds can reactivate and start to produce new shoots.
\begin{figure}[hbt]
    \centering
    \includegraphics[width=4.5in]{figs/image1.jpg}
    \caption{a) Plant modules of a self-organizing tree growth
model, b) shadow space of a tree in used tree growth model from \cite{kohek_eduapple:_2015}.}
    \label{fig:my_figure1}
\end{figure}


\subsection{Illumination}

We calculated the inner shading by using the algorithm from \cite{palubicki_self-organizing_2009}
that was later extended in~\cite{pirk_plastic_2012,stava_inverse_2014,strnad_novel_2017}. The light distribution
inside the tree crown is calculated from the illuminated leaves that are
sources of a conical shadow volume. We calculate the bud illumination.
Suppose, we have a bud inside one of the shadow volumes of one of the
neighbors. Let \(d_{y}\) be the bud's vertical distance from the volume
apex and \(d_{x,z}\) its horizontal distance from the volume axis, then
the received shadow, originating from a given shadow volume is
calculated by \cite{strnad_novel_2017}:
\begin{equation}
\Delta s = \left\{ \begin{matrix}
\text{ab}^{- 0.8\left( d_{y} + d_{x,y} \right)}, & \mathrm{\text{if}}\ d_{x,y} < d_{y} \\
0, & \mathrm{otherwise} \\
\end{matrix}, \right\},
\end{equation}
where \(a\) \textgreater{} 0 and \(b\)\textgreater{} 0 are model
parameters from \cite{palubicki_self-organizing_2009} and were set to \(a = 0.05\) and \(b = 2\).
Total irradiance of a given bud is then calculated by:
\begin{equation}
  Q = max\begin{Bmatrix}
1 - s, & 0 \\
\end{Bmatrix},  
\end{equation}
where \(s\) is the cumulative contribution of all shadow volumes,
captured by the given bud.

\subsection{Differential Evolution and Pruning}
Differential Evolution (DE) is a heuristic approach for minimizing
possibly nonlinear and non-differential continuous objective function.
Developed by Storm and Price \cite{storn_differential_1997}, the method aims to optimize
certain properties of a system pertinently choosing the system
parameters, usually represented as a vector. Objective function models
the objectives while incorporating potential constraints. DE employs
evolutionary operators like mutation, crossover, and selection.

Discrete DE (DDE) variants have been presented in relation to specific
combinatorial optimization problems, e.g., \cite{davendra_flow_2009,pan_discrete_2008,wang_novel_2010}. The DDE has
been used for the optimization of light condition inside the tree crown
by removing certain branches in \cite{strnad_novel_2017} in particular four optimization
methods have been used: BinDE, IndexDE, PathDE, and SetDE. Through
optimization of pruning locations, a combination of cuts is obtained
that maximizes the amount of light received by the remaining buds of the
tree crown. For that purpose, for all buds in the tree crown the
irradiance is calculated by the use of Eq. (2). Bud's irradiance, \(Q\)
corresponds to the percentage of available light intercepted by the bud.
For the sake of faster detection of ancestor/successor relationship
between internodes, each internode is associated with a unique
variable-length binary string as shown in~Fig.~\ref{fig:my_figure1}~b. Tree crown light
distribution is calculated next, where the irradiance of each bud in the
crown is assigned to one of ten quantization classes of equal width on
the interval {[}0, 1{]}. The objective function, \emph{f}(\textbf{x}),
is:
\begin{equation}
 f\left( \mathbf{x} \right) = \frac{\sqrt{S_{\mathrm{\text{tree}}}}}{H}\sum_{i = 1}^{10}{i \times h_{i}}, 
\end{equation}
where \(S_{\mathrm{\text{tree}}}\) is the number of remaining internodes
after pruning, \emph{H} is the total number of buds, and \(h_{i}\) is
the number of buds in the \emph{i}-th class of light distribution.

We define an optimization function that describes the light interception
of the tree as the sum of interception of all buds
\begin{equation}
   f\left( \mathbf{x} \right) = \sum_{i = 1}^{n}E_{i},
\end{equation}
where $n$ is the number of buds that remain after the pruning,
$E_{i}$ is the irradiance of the \(i\)-th bud. The function $f(\mathbf{x})$ is the total irradiance of the tree for a configuration \(x\) of the buds,
also called the solution vector. The solution vector is denoted by
\(\mathbf{x} = \ \left\{ x_{1},\ \ldots,x_{s} \right\}\) and it contains
the encoded sequence of cut positions, labeled by the corresponding
internodes. The root is denoted by a bit-string ``0''. A `0' or `1' is
appended to the parent's string for each main or lateral child
internode, respectively (Fig.~\ref{fig:my_figure1}~b). Each cut position \(x_{i}\)
identifies the internode, at which the branch is removed. Variable
\emph{s} in this case is a population size and is a value between
\(s_{\mathrm{\min}}\) and \(s_{\mathrm{\max}}\), which are custom set
parameters, representing the minimum and the maximum number of allowed
cuts. The objective function \(f(\mathbf{x})\) favors solutions that
provide a maximal improvement of light distribution with minimum amount
of removed biomass. It is very important to recognize a redundant cuts,
for example, if a cut should be made at internode ``00'', the cuts
``001'', ``000'', or ``0010'', would be redundant and thus unnecessary
as shown in Fig.~\ref{fig:my_figure2}. Our labeling allows for quick detection of the
redundant cuts. In a given example the later three internodes share the
same prefix ``00'', which is the label of their predecessor and with
removing it, all of its successors are removed as well.

\begin{figure}[hbt]
    \centering
    \includegraphics[width=3.3in]{figs/image2.jpeg}
    \caption{Mapping of the solution vertex into the tree cutting
sequence. Solution vector components that exceed the threshold are
converted into the cuts with the help of the cut list.}
    \label{fig:my_figure2}
\end{figure}


For our experiment, we have selected only BinDE optimization method
because the pruning results were visually close to those of the human
expert. The encoding scheme of BinDE uses the real-valued vectors to
represent solution genotypes. For the mapping of genotypes to pruning
instances, the intermediate cut list is used. It is produced by
traveling the tree model in a depth-first manner and recording the
labels of all internodes encountered in the process into a list. The
length of the cut list defines the dimensions of solution vectors. Each
component \(x_{i,j} \in \ \left\lbrack 0,\ 1 \right\rbrack\) of the
solution vector \(\mathbf{x}_{i}\ \)determines whether \emph{j}-th
internode from the cut list will be selected as the cutting point or
not. For that purpose the binary vector \(\mathbf{z}_{i}\) is
constructed by thresholding:
\begin{equation}
    z_{i,j} = \ \left\{ \begin{matrix}
0, & \mathrm{\text{if}}\ x_{i,j} < 0.5 \\
1, & \mathrm{\text{otherwise}} \\
\end{matrix}, \right.\
\end{equation}


If the number of cuts proposed by the vector \(\mathbf{z}_{i}\) violates
the solution size constraints, the threshold is adjusted up or down
until the constraints are met. The entire mapping operation is shown in
Fig.~\ref{fig:my_figure2}.

\subsection{Tree Height and Neighboring Distance Control}
Although the DDE method significantly improves the light conditions
inside the tree crown, it offers no control over the tree size or the
distance to neighbor trees. Since the modern orchards are mostly
protected by the anti-hail nets, it is crucial to keep apple trees at a
certain height. Similarly, the neighboring distance that, has to be
preserved during the entire lifetime of an orchard.

A straightforward method to get control over the tree height and
neighboring distance would be the extension of the objective function
\(f\left( \mathbf{x} \right)\) with additional constraints regarding the
tree size. It turned out, however, that while the integration of the
tree height into the tree model and objective function is possible, the
determination of the extent of the tree crown after the pruning was not
very efficient.

Our inspiration for the solution to this problem comes from observing a
human during manual pruning. Instead of just following the pruning
rules, the human pruner strives to shape the tree into one of the
well-defined growing forms. Those growing forms were developed over the
years by the experts and gave the best fruiting results under certain
growing conditions. In the high density orchards, the most common tree
forms are the Slender Spindle \cite{weber_optimizing_2000}, and more recent, the Tall
Spindle \cite{robinson_vision_2013}. While the first growing form is more conically shaped
the second resembles a cylinder. Tall spindle is very popular because it
is suitable for mechanical pruning and formation of the Fruiting Wall
\cite{robinson_vision_2013}.

Instead of changing the objective function, we propose adding a
preprocessing level to the DDE method. We call this step pre-pruning and
by adding this the desired form, the tree height and neighboring
distance can be controlled easily. We prune the trees to circular shapes
(cone and cylinder) with adjustable height and the base radius. After
the branches outside the chosen shape are trimmed off, the DDE method is
used to optimize the light condition inside the tree crown. The entire
tree pruning process is depicted in Fig.~\ref{fig:my_figure3}.
\begin{figure}[hbt]
    \centering
    \includegraphics[width=5.3in]{figs/image3.jpeg}
    \caption{Two-step virtual pruning process enables the control
over tree height and neighboring distance. First, the tree is pre-pruned
into a cone or cylinder shape with adjustable size. In the second step
the DDE method selectively removes branches to improve the light
conditions inside the tree crown.}
    \label{fig:my_figure3}
\end{figure}

The results of the final pruning are highly dependent on the maximum
allowed number of cuts \(s_{\mathrm{\max}}\), which has to be adjusted
to the tree age and complexity of the tree crown. In our implementation,
we set \(s_{\mathrm{\max}} = 20\) which provided good results for young
trees. While we used circular proxies in our work, other shapes could be
used: for example, orchards with Fruiting Wall planting system could use
rectangular blocks.

\section{Results}
By using our proposed method, we pruned a four years old virtual
untrained tree (shown in Fig.~\ref{fig:my_figure4}) by using the proposed method and
compared the results to the pruning performed by horticulture expert.
The tree training teaching environment EduAPPLE \cite{kohek_eduapple:_2015} was used for
this purpose. The expert shaped the tree into two primary forms, a
pyramidal growing form HeP, which is similar to a Slender Spindle, and a
growing form HeF, suitable for the Fruiting Wall planting system. For
the automated pruning, we used a cone with the height of \(3\)m and the
opening angle of \(45{^\circ}\) (DDECn), and a cylinder with a height of
3m and radius of 0.75 m (DDECy). In both cases, \(s_{\mathrm{\max}}\)
was set to 50. The pruning results can be seen in Fig.~\ref{fig:my_figure4}.
\begin{figure}[hbt]
    \centering
    \includegraphics[width=5.4in]{figs/image4.jpeg}
    \caption{Comparison of pruning with the initial unpruned tree,
tree pruned by a human expert in a pyramidal shape (HeP), automated
pruning with initial cone shape (DDECn), pruning by human expert pruning
in a plat plane (HeF), automated pruning with the use of cylindrical
initial shape (DDECy).}
    \label{fig:my_figure4}
\end{figure}

In all four cases, the light distribution inside the tree crown is
significantly improved as compared to the unpruned tree (Fig.~\ref{fig:my_figure5}). In the unpruned
tree only 14.89 \% of buds receive more than 70 \% of available light while in the the human-pruned tree the amount of such buds increases to 26.80 \% (HeP) and to the 22.95 \% (HeF).
The light distributions of automated tree pruning are comparable with
that of human expert (25,93 \% for DDECn and 24.18 \% for DDECy). By comparing HeP to DDECn the result in the cases of human pruning is slightly better but in the case of HeF and DDECy the automated pruning achieved better result, althoug the difference in both casses is less than 1.3 \%. The difference between the DDE
and human pruning is also visible in the category of the internodes left
in the tree after the pruning.

By combining both results, it can be concluded, that the human expert
achieved similar bud light exposure with less biomass removed,
which is desirable since the higher number of internodes with
well-illuminated buds represent the higher potential for the yield of
both increased quality and quantity. Another metric is to compare the
percentage of buds in the tree crown receiving more than 40 \% of light
available. In this case, the difference between automated and human
pruning is 0.87 \%, in the case of HeP and DDECn and 6.86 \% by methods HeL and DDECy in the favour of the later.

\begin{figure}[hbt]
    \centering
    \includegraphics[width=5in]{figs/image5.jpeg}
    \caption{Evaluation of tree pruning results, a) Comparison of
light distribution after the pruning by a human expert (HeP and HeF) and
automated pruning (DDECn and DDECy), b) Number of internodes left after
the pruning. A higher number of internodes, combined with higher light
exposure of buds signify better results.}
\label{fig:my_figure5}
\end{figure}

DDECn and DDECy are representative of the selective pruning, where the
results of the first step represent the result of pruning by the
automatic pruning systems currently used. The difference between the
first and second step of DDECn and DDECy can be seen in Fig.~\ref{fig:my_figure6}.

\begin{figure}[hbt]
    \centering
    \includegraphics[width=5.4in]{figs/image6.jpeg}
    \caption{Tree crown light distribution after the first and
second pruning step, a) DDECn method, b) DDECy method}
    \label{fig:my_figure6}
\end{figure}

We also wanted to evaluate a long-term exposure to pruning and we were
curious if the proposed automated tree pruning method is capable of tree
training into desired growing form without human intervention. We have
simulated a row of five trees for six consecutive years. At the
beginning of each year, the trees were pruned by DDECn method to shape
them into the Slender Spindle growing form. The starting cone height was
1m and was linearly increased to 2.5m, which was the target height for
the trees in the following three years. The opening angle was constant
at 45° for the experiment duration. The initial value of the parameter
\(s_{\mathrm{\max}}\) was set to 20 and was linearly increased to 70 in
the sixth year. The result of the experiment can be seen in Fig.~\ref{fig:my_figure7}.

\begin{figure}[hbt]
    \centering
    \includegraphics[width=3.84333in,height=4.66333in]{figs/image7.jpeg}
    \caption{Tree training of five apple trees into a Slender
Spindle growing form for six consecutive years with the DDECn method.
The purple color denotes removed branches}
    \label{fig:my_figure7}
\end{figure}


In general, as the tree structure becomes more complex in time, so the
value of \(s_{\mathrm{\max}}\ \)has to be increased. In our experiments,
we determined, that \(s_{\mathrm{\max}}\ \)should not exceed 150 even
for older trees.

\section{Discussion}

We have introduced an automated method for simulation of pruning of
trees and tree colonies. The objective was to propose pruning that
maximizes light exposure of buds within the crown and we used two step
method, where first step prunes the tree to a desired shape, and the
second step maximizes the bud irradiance. Our results show that this
combination comes close to the human pruning regarding the light
distribution inside the tree crown. The pruning simulation of a group of
trees for several consecutive years showed that the method could also be
used for the tree training towards desirable growing form. Since DDE is
a stochastic method, the set of excess branches that have to be removed
can vary from simulation to simulation. The solutions are not always
equally successful with respect to tree light crown distribution, but
the same is also true by the pruning solutions proposed by human pruning
expert.

The possible limitation of the proposed method is its strong dependency
on the tree growth model topological structure. This is important
because the challenging step in the automated tree pruning is the
construction of the appropriate tree model from tree images. Successful
algorithms of this kind are presented in {[}1, 5, 10{]}. However, they
are not directly compatible with the EduAPPLE tree growth model and a
reconstruction algorithm should be developed for that purpose. Since the
proposed pruning method shows pruning results comparable to human
expert, the construction of such an algorithm would be desirable. Some
work in this direction has already been done \cite{kohek_estimation_2017}, but it is too
early to assess its efficiency.

The main contribution of our work is to show that good pruning results
can be obtained automatically and without a fixed set of pruning rules.
This is not surprising since the pruning rules were developed by the
long-term experiments whose goal was to improve yield quality as well as
its quantity. Once the main parameters that influence the yield were
determined, the pruning techniques have been developed that maximize the
influence of those parameters and that is what the DDE objective
function is trying to model. Since the light exposure is a critical
factor, the proposed method searched for the combination of cuts that
maximize the light exposure. This resulted in the higher light
distribution inside the tree crown. This distribution was also used for
evaluation of tree pruning since the obtained tree forms cannot be
compared directly. While the shapes of the trees in Fig.~\ref{fig:my_figure4} differs much
their light distributions are comparable. The additional pruning
benchmark is the number of internodes, which represents the equivalent
to the tree volume. The amount of removed internodes corresponds to the
pruning intensity. Fig.~\ref{fig:my_figure5}~b shows that the human expert preferred
slightly less aggressive pruning than the DDE method. The pruning
intensity by the proposed automated pruning is controlled by the
relation between tree volume and the value of the \(s_{\mathrm{\max}}\)
parameter. In order to preserve the tree pruning intensity, as the tree
is growing, the value of \(s_{\mathrm{\max}}\) have to increase with the
increasing number of internodes in the tree. In our experiments, 20
turned out to be the good initial value for the \(s_{\mathrm{\max}}\)
parameter.

\section{Conclusion}

Dormant tree pruning is very labor and cost intensive process, which is
essential for achieving good fruit yield and for the health of the fruit
trees. It is not surprising that extensive research has been done on how
to automate this process. Early experiments concentrated on the mass
pruning, which remains the most widespread automated pruning method to
this day. Although those methods produce good results on the Fruit Wall
planting system, they are not equally successful on other planting
systems, where more selective pruning is needed. In this article, we
presented a two-step method suitable for automated selective pruning.
Experimental pruning of virtual trees has shown, that the proposed
system is capable of achieving similar light condition in the three
crown as a human expert. The method is able to control the tree height
and the neighboring distance. In the performed simulations, it was
possible to shape the trees into desired growing form autonomously.
According to the results, the method has great potential. In order to be
used on real trees, some of technological problems have to be solved
first among which is the reliable reconstruction of a tree model out of
the set of photographs.

\section*{Acknowledgment}

The authors acknowledge the Project BI-US/17-18-012, and Programme
P2-0041 were supported financially by the Slovenian Research Agency.

\bibliography{Bilateral}

\end{document}
